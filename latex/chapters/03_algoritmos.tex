\chapter{Descrição em Pseudocódigo dos Algoritmos da API}
\label{ch:pseudocode}

Neste capítulo, apresentaremos a descrição dos algoritmos, na forma de pseudocódigos, que constam na especificação da API. Este capítulo tem por objetivo preparar o leitor para ler e compreender com clareza a implementação feita em Rust, trazendo descrições e explicações que elucidem o funcionamento de cada algoritmo.

\section{Algoritmos de Árvore Geradora Mínima}
\section{Algoritmos de Caminho Mais Curto}

\subsection{Algoritmo de Dijkstra}
O Algoritmo de Dijkstra \cite{dijkstraarticle}, proposto pelo cientista da computação Edsger Dijkstra, é um algoritmo extremamente útil para encontrar o caminho mais curto dentro de um grafo ponderado orientado ou não orientado sem arestas negativas. 

\begin{algorithm}
	\caption{Algoritmo de Dijkstra}
	\begin{algorithmic}[0]
		\Require{$ \mathbf{G(V, A, W)}$, $v \in \mathbf{V}$}
		\Ensure{Sequência de $v \in \mathbf{V}$ ordenados que formam o caminho mais curto}
		\Statex
		\Function{Dijkstra}{G (V, A, W), s}
		\State{$predecessor \gets []$}
		\State{$predecessor[s] \gets$ nulo}
		\State{$visitado \gets []$}
		\State{$visitado[s] \gets 1$}
		\State{$distancia \gets []$}
		\State{$distancia[s] \gets 0$}
		\For{$v \in V$}
		\If{$v \in G.vizinhos()$}
			\State{$predecessor[v] \gets s$}
			\State{$distancia[v] \gets w(sv)$}
		\Else
			\State{$predecessor[v] \gets$ nulo}
			\State{$distancia[v] \gets \inf$}
		\EndIf
		\EndFor
		\While{$u \in V \land u \not \in visitado$} 
		\State{$v \gets \min V$}
		\State{$visitado[v] \gets 1$}
		\For{$n \in v.vizinhos()$}
		\If{$n \not \in visitado \land distancia[n] > distancia[v] + w(vn)$}
			\State{$distancia[n] \gets distancia[v] + w(vn)$}
			\State{$predecessor[n] \gets v$}
		\EndIf
		\EndFor
		\EndWhile
		\State \Return{$(visitado, distancia)$}
		\EndFunction
	\end{algorithmic}
\end{algorithm}

\section{Algoritmos em Grafos Eulerianos}
